

\chapter{Introduction}
\label{chap:Introduction}

\section{NASA Magnetospheric MultiScale Mission}
\label{sec:NASAMagnetosphericMultiScaleMission}

NASA's Magnetospheric MultiScale (MMS) Mission is classified as a Solar Terrestrial probe mission and is scheduled for launch in October 2014 \cite{mms_website}.  The mission consists of four identical satellites orbiting the Earth in a constellation formation flight.  The satellites are being constructed at NASA's Goddard Space Flight Center with the purpose of studying the microphysics of magnetic reconnection within the Earth's magnetic fields.

\TODO{image of MMS}

Reconnection events are caused when electromagnetic energy from the sun interacts with the Earth's magnetosphere causing magnetic field lines to cross.  Adjacent magnetic fields generally have significantly different orientations so when a reconnection event occurs, a large quantity of energy is released in the form of heat and kinetic energy.  The most widely known consequence of a reconnection event is the northern lights.

Despite the clearly visible effects of reconnection events, very little is known about the microphysics inside its the diffusion region.  Magnetometers, spectrometers and other equipment currently in orbit are only able to capture a small fraction of the event's behavior.  Most equipment collect data from a single point or direction in space or some can get a 360 view of space by applying a slow spin to the spacecraft.  In both cases, the reconnection event can pass by at 10-100 km/s which is much too fast for even a spinning satellite to obtain a full view of the diffusion region.

MMS's satellites, when deployed, will be equipped with instrumentation mounted at the end of six boom extending out from the spacecraft's body along each major axis.  Four boom are the Spin Plane Double Probes (SDP) and two Axial Double Probes (ADP).  This configuration gives each of the four satellite six distinct points to capture data about the diffusion region as it passes.  That information from all the sensors can then get combined to form a 3D representation of the diffusion region.

Since the satellites are spin stabilized and have these large flexible ADP and SDP booms, tight control on the body's attitude is required.  Inaccuracies in the satellite's attitude can introduce errors in the position of the ADP and SDP instrumentation.

% Science questions to answer \cite{mms_website}
%     What determines when reconnection starts and how fast it proceeds?
%     What is the structure of the diffusion region?
%     How do the plasmas and magnetic fields disconnect and reconnect in the diffusion regions?
%     What role do the electrons play in facilitating reconnection?
%     What is the role of turbulence in the reconnection process?
%     How does reconnection lead to the acceleration of particles to high energies?


% \todo{image of formation flight}
% study microphysics of
%   magnetic reconnection
%   energetic particle acceleration
%   turbulence

% s/c MMS-1, MMS-2..MMS-4
% reconnection: Electromagnetic engergy from the sun interacts with Earth's magnetosphere ausing magneti field lines to cross and create a burst of energy \cite{nasa_edge_video_ne_at_mms}
% magnetic reconnection measured ions and electrons as boundary passes to create 3d model of it passing by
% Fast plasma investigation
% probing the electron diffusion region (EDR) (passes too rapidly to get an accurate view with current equipment small (1-10 km) and rapidly moving (10-100 km/s))
% adjacent magnetic fields generally have significaltly different orientations such that when they intersect, a large amount of energy is dispursed within a small region in the form of heat and kinetic energy.  The region = diffuision region
% explore magnetic reconnection
% dynamic regions of magnetosphere
% orbits planned to pass through the upstream and downstream magnetic reconnection sites
% 1) day side - solar wind field lines connection
% 2) down stream -
% 3) plasma travels down and causes the Arura
% through magnetospheric reconnection, portals allow energetic particles to traverse from outside to the interior of the magnetosphere
% predict when solar space weather within the magnetosphere and if they will affect orbiting satellites
% adverse space weather within the magnetosphere can negatively impace spacecraft system health GPS, induce disruptive current in electrical grids, communictaions, increased radiation exposure on trans-polar flights

% Difficult to understand
% magnetic boundary passes satellites quickly so has been hard to measure
% MMS has intsruments to capture measurements
% Instruments mounted on all sides of the
% 8 sensors 1/30th sec instead of

% October 2014, Atlas five launch \cite{nasa_edge_video}

% Sensors
%   star sensor -> attitude
%   accelerometers -> $\Delta V$


% Fast Plasma Instrument (FPI) - controls
%   16 Dual Electron spectrometer (DES) Goddary built
%   16 Dual Ion Spectrometer (DIS) Japan *** built, hand delivered
%   180 degree and +- 22degree measurement
%   30 millisec measurement rate 100x faster than previous missions entire view of sky
% FPI
%   despins data

% Instument Data processing Unit (IDPU) - brains of measurements
%   collects, compresses, transmits requested measurement data
%   configure while on mission

% booms
%   eight deployable booms
%   two 12.5m axial booms (electric field sensors)
%   four 60m wire booms
%   two 5m booms in spin plane for magnetometers
% rigid, wire, top/bottom booms
% important to keep consistent spin rate to get accurate estimates of boom location


% \section{Propulsion}

% types: solid propellents, bi propellents, electro propulsion, cold gas systems
% chose: mono-propellant blowdown - hydrozene power thrusters \cite{nasa_edge_video_propulsion}
% 3 rpm
% radial thrusters - spinning
% axial thrusters - prevent nutation

% concern:
% propulsion introduce distrubances

% 20 second pulses

% fuel limits by number of adjustments


% \section{performance requirements}

% $\pm 0.5$ deg attitude tollerance
% 1/10th of a second

\section{Research Objective}
\label{sec:ResearchObjective}


\section{Thesis Outline}
\label{sec:ThesisOutline}


